\documentclass[12pt, oneside]{article}
\usepackage{a4wide}
\usepackage{oldgerm}
\usepackage{amsmath}
\usepackage{amssymb}
\usepackage{amstext}
\setlength{\textheight}{8.875in} \setlength{\textwidth}{6.875in}
\setlength{\columnsep}{0.3125in} \setlength{\topmargin}{0in}
\setlength{\headheight}{0in} \setlength{\headsep}{0in}
\setlength{\parindent}{1pc} \setlength{\oddsidemargin}{-.304in}
\setlength{\evensidemargin}{-.304in}



\begin{document}
\setlength{\textheight}{8.5in}
\centering {\bf MTL 106 (Introduction to Probability Theory and Stochastic Processes) }\\


\centering{\bf Assignment 2 Report}



\vskip 0.5cm

\noindent Name: Arpit Saxena ~~~~~~~~~~~~~~~~~~~~~ Entry Number: 2018MT10742



\vskip 0.5cm



\begin{enumerate}
	




\item Basic Probability

\item Random Variable/Function of a Random Variable



\item Stochastic Processes

\item Stochastic Processes

\item DTMC


\item DTMC

\item CTMC


\item CTMC


\item {
    Queueing Models

    A company has one 16-core machine, two 8-core machines and two 4-core machines. They want
    to use them as servers. The inter arrival time of queries is exponentially distributed
    with mean 0.1ms. They estimate that the time taken by a core per query
    would be exponentially distributed with mean time 3, 2, 4 milliseconds for the 16-core,
    8-core and 4-core machines respectively. They want to set up a simple static load 
    balancer in front of these machines, which will schedule the queries on a core of a 
    machine with a probability to assign the query to each core.

    Determine the probabilities by which the load balancer should schedule queries on each
    type of core to minimise the maximum expected waiting time for a query. Note that one 
    query would occupy the core on which it's running for the entire time it's running.

    \textbf{Answer}

    Basically, what we want to do here is to divide the incoming queries into 3 different
    types of cores (they are different since they have different service time distributions)
    We can tabulate the information as follows:

    \begin{center}
        \begin{tabular}{| c | c | c | c |}
            \hline
            Number of machines & Number of cores & Total number of cores & Mean service time(ms) \\
            \hline
            1 & 16 & 16 & 3 \\
            2 & 8 & 16 & 2 \\
            2 & 4 & 8 & 4 \\
            \hline
        \end{tabular}
    \end{center}

    Number the types of cores given in the above table as 1, 2, 3 and let \(p_1, p_2, p_3\)
    denote the probabilities by which a query will be sent to a core of that type by the load
    balancer.

    Given that the incoming queries form a Poisson process with parameter 
    \(\frac{1}{0.1 \text{ ms }} = 10 \text{ ms}^{-1}\) and the load balancer is decomposing
    this Poisson process into separate streams. Then, the queries going to cores of type
    1, 2, 3 will form a Poisson process with parameters \(10p_1, 10p_2, 10p_3\) respectively
    and \(16p_1 + 16p_2 + 8p_3 = 1\) since each query will be routed to one of the given
    cores.

    We now model each core as a M/M/1 queue. Generically, let's take the arrival process
    parameter as \(\lambda\) and the service time parameter as \(\mu\).

    Using the result derived in class, let \(W\) be the waiting time for a query, then
    its CDF in the steady state is given as
    \[
        P(W \leq t) = \begin{cases}
                        0 & t < 0 \\
                        1 - \rho & t = 0 \\
                        1 - \rho e^{-(\mu - \lambda)t} & 0 < t < \infty \\
                      \end{cases}
    \]
    where \(\rho = \frac{\lambda}{\mu}\) and the steady state solution is only possible
    when \(\rho < 1 \implies \lambda < \mu\)

    Then the pdf of \(W\) is given by \(f_W(t) = \rho(\mu - \lambda)e^{-(\mu - \lambda)t}\) when
    \(0 < t < \infty\) and 0 otherwise.
    \begin{align*}
        E(W) &= \int_0^\infty t \rho(\mu - \lambda)e^{-(\mu - \lambda)t} \,dt \\
             &= \frac{\rho}{\mu - \lambda} \int_0^\infty e^{-(\mu - \lambda)t} (\mu - \lambda) t \,d[(\mu - \lambda)t] \\
        \intertext{Since \(\mu - \lambda > 0\), \((\mu - \lambda)t \to \infty\) as \(t \to \infty\)}
        E(W) &= \frac{\rho}{\mu - \lambda} \int_0^\infty te^{-t}\,dt \\
             &= \frac{\rho}{\mu - \lambda} \\
             &= \frac{\lambda}{\mu^2 - \lambda\mu} \tag*{\(\left(\text{Since } \rho = \frac{\lambda}{\mu}\right)\)}
    \end{align*}

    Now substituting the values of \(\mu\) as \(\frac{1}{3}, \frac{1}{2}, \frac{1}{4}\) for
    cores 1, 2, 3 respectively and also using the query process parameters as calculated above,
    we get the expected waiting times for cores 1, 2, 3 respectively as:
    \[
        \frac{90p_1}{1 - 30p_1}, \frac{40p_2}{1 - 20p_2}, \frac{160p_3}{1 - 40p_3}
    \]
    
    We also need to have \(\rho < 1\) for all the cores, i.e. \(\frac{10p_i}{\mu_i} < 1\) 
    for all the cores, which gives \(p_1 < \frac{1}{30}, p_2 < \frac{1}{20}, p_3 < \frac{1}{40}\)

    So our problem reduces to the following optimisation problem:
    \begin{align*}
        \text{Minimise } &max\left\{\frac{90p_1}{1 - 30p_1}, \frac{40p_2}{1 - 20p_2}, \frac{160p_3}{1 - 40p_3}\right\} \\
        \text{In the domain } &16p_1 + 16p_2 + 8p_3 = 1, p_1 < \frac{1}{30}, p_2 < \frac{1}{20}, p_3 < \frac{1}{40}
    \end{align*}

    Solving the equations with the aid of computational tools available, we find that the minimum expected waiting time
    is approximately 4.77 ms, when \(p_1 \approx 0.020, p_2 \approx 0.035, p_3 \approx 0.013\)

    So, we have the probabilities by which the load balancer should send the queries to
    cores of type 1 as \(16p_1 \approx 0.327\), cores of type 2 as \(16p_2 \approx 0.564\)
    and cores of type 3 as \(8p_3 \approx 0.109\) for minimisation of the maximum expected
    waiting time for each query.
}

\item Queueing Models

\end{enumerate}

\end{document}
